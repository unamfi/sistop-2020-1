\documentclass[12pt]{article}

\usepackage[letterpaper,right=2cm,left=2cm]{geometry}
\usepackage[utf8]{inputenc}
\usepackage[spanish]{babel}
\usepackage{graphicx}
\usepackage{hyperref}
\usepackage{fancyhdr}
\usepackage{sans}
\usepackage{sansmathfonts}
\usepackage{multicol}

\pagestyle{fancy}

\begin{document}

\begin{titlepage}
  \begin{center}
    \Huge{Universidad Nacional Autónoma de México}
    \vfill
    \includegraphics[width=0.2\linewidth]{img/UNAM_INGENIERIA}
    \vfill
    \LARGE{\emph{Facultad de Ingeniería}}
    \vfill
    \LARGE{\texttt{Sistemas Operativos}}
    \vfill
    \Large{Alumno: Romero Andrade Cristian\\Número de Cuenta: 313160282} 
    
    \vfill
    
    \huge{Proyecto 1\\Reseña: Mi Computer $20^o$ Fascículo}
    \vfill
    
    
    \large{Profesor: Gunnar Eyal Wolf Iszaevich}

    \vfill

    {\large \today\par}
    \newpage
  \end{center}
\end{titlepage}

% \section{Resumen}

%\tableofcontents\newpag

%\section{Introducción}\label{sec:introduction}

El presente consiste sobre una revista famosa de los antaños 80's\footnote{Abril 1984} \texttt{Mi Computer},
la versión de habla hispana de la revista \texttt{Home Computer Course} la cual contiene artículos sobre
diversos temas, como son el hardware, software, programas informáticos, programación en \textit{BASIC} y el
tema que siempre destaca en la revista, la revisión y análisis de un \textit{microordenador contemporáneo} donde
exponen imágenes de sus partes internas de manera despiezada.\\

En este fascículo toma en su sección \textit{Modelos de Hardware} la \textbf{Memotech MTX 512}, un computadora
doméstica de arquitectura \textit{Z80} lanzada por \textit{Memotech} entre 1983 y 1984, la maquina era
demasiado atractiva comparándolas con la competencia de la época. La forma en la que presenta la máquina es
llamativa tanto para el lector casual como el lector recurrente, ya que visualmente se ve el diagrama de
despiece, la máquina destaca por su facilidad de desarmarla, darle mantenimiento, y
expandirla intercambiando los componentes
que venían por defecto. También destaca que el ordenador tenia tantos chips le cabían y minimizar el uso de las
ULAs\footnote{Uncommited Logic Array}, ya que los diseñadores lo prefirieron por cuestiones estéticas
o por que así salia más económico y mantenible.

En su diagrama se señalan de lo que esta constituida la maquina, la RAM de $64[KB]$ expandible a $512[KB]$,
las interfaces de entrada y salida para las
impresoras, Cassetes Joysticks\footnote{Compatibles con la norma Atari}, y, lo que un audiofilo puede considerar
el uso de esta bella maquina es que esta tenia un conector de alta fidelidad (Hi-Fi) permitiendo poder
conectar un altavoz que proporciona un sonido de mayor calidad, lo cual era extraño para los ordenadores
personales de esos tiempos.

Esta computadora personal contenía lenguajes de programación previamente ya suministrados, siendo uno de
ellos \texttt{NODDY}, conteniendo solo 11 órdenes siendo sencillo para usuarios con necesidad de manejar texto
ya que este lenguaje no podía efectuar operaciones aritméticas. Igualmente contenía el famoso lenguaje
\texttt{BASIC} y la capacidad de utilizar el lenguaje \texttt{Ensamblador}\label{asm}

Por las especificaciones dadas vemos que es una máquina completa con una apariencia que supera a varios
ordenadores de la época, esta estaba diseñada para que sea fácil de desarmar y tener  acceso al interior.
Se da una observación a que su placa base tiene un gran número de circuitos integrados pero sin la
utilización de algunas \textit{ULA's}\footnote{Uncommited Logic Array} de mayor capacidad en la época,
haciendo que el mantenimiento correctivo del computador sea más eficiente, ya que las \textit{ULA's}
dificultan la tarea anterior.\\

En lo anterior explicado se menciona el uso escaso de ULAs en la \textbf{Memotech MTX 512}, en este fascículo
se tiene una sección llamada \textit{Términos Clave} que define conceptos o herramientas comunes en la
informática, este primer apartado de \textit{Términos Clave} define  lo que es una ULA, cómo se constituyen
y como se diseñan, siendo estas circuitos integrados que constan de un gran número de compuertas lógicas no
comprometidas se consideran un desarrollo de la ROM ya que el contenido de ambas es determinada por el
fabricante y no por el usuario. Refiriéndose como \textit{módulos}, estas tienen una gran variedad de
interconexiones, reduciendo la cantidad de chips de ordenadores complejos.

Más adelante después de otras cuantas secciones nos encontramos de nuevo con \textit{Términos Clave}
Donde esta habla del funcionamiento de los ``Programas que corrijan programas generados por el humano''
Llegando a concluir de lo que se va a hablar es de un compilador pero no de cualquier compilador que
avisa en donde el programador se equivoco, sino uno que corrige el error del programa por si mismo
y de como se conseguía. Inmediatamente nos hablan de como obtener el programa arregla-programas, estas
se conseguían comprando chips de ROM suplementarios o cartuchos de software que ampliaban el set de ordenes
disponibles en el lenguaje \texttt{BASIC} para así eliminar a su mayor alcance los errores que aparecían
en los programas.

Estas ``expansiones'' de ordenes lo consideraban como ``el juego de herramientas
del programador'' que se encuentran en el mercado de chips más cercano. Aunque esta idea se abandonaba
poco a poco ya que la investigaciones señalaban a sistemas desarrollados para \textit{crear programas}
en vez de \textit{corregir} los existentes.

Aunque suene que los \textit{generadores de programas} el autor plantea que en un futuro inmediato,
y hasta la fecha sigue siendo cierta, estas no serán capaces de sustituir al programador humano, para
ello da tres razones, la primera es la técnica que el humano tiene para aplicaciones comerciales, contables,
control de inventario o en lo considerado arte, en la creación de videojuegos. Como segunda razón, el
generador de programas es incapaz de hacer eficiente un programa ya que este se tiene que obligar con
las reglas del S.O. y el uso de memoria, lo cual un programador humano si es capaz de hacer. Y en tercer lugar,
solo un humano puede entender la necesidades de otro humano para que la interacción del programa sea
cómoda.

A medida en que se transita por sus páginas se evidencia la complejidad de la informática, pero
\textit{Mi Computer} hace en cada fascículo un apartado donde habla del lenguaje \texttt{BASIC}, que manejan
esta sección como un mini-curso, hablando de su colección de instrucciones que el lenguaje proporciona,
eso si no se tiene experiencia en programación o no se han seguido estos ``cursos'' desde fascículos
anteriores, será un poco complicado entender a la primera de lo que están hablando.\\

Aunque suenen pesados estos temas, la revista hace que la lectura sea atractiva por la cantidad de
ilustraciones y de como esta redactado sin dejar perder el punto ``técnico'' a lo que quieren hacer
llegar, expandir el conocimiento computacional en diferentes aspectos como en la industria de los videojuegos,
en la ficción del cine, en la informática y en el hardware.


% \begin{multicols}{2}

% \section{Modelo de Hardware\ -\ Memotech MTX 512}

% Toda revista \texttt{MyComputer} tiene esta sección donde hablan de una computadora, exhibiendo sus partes y
% componentes, la computadora de la que habla específicamente este fascículo es la \textbf{Memotech MTX 512},
% un computador de arquitectura \textit{Z80}. Empezaremos mostrando la lista de especificaciones que nos presenta
% la revista.

% \begin{table}[h]
%   \centering
%   \begin{tabular}{|c||c|}
%     \hline
%     Dimensiones&$488x202x56[mm]$\\\hline
%     CPU&Z80\\\hline
%     Reloj&4 [MHz]\\\hline
%     Memoria&ROM: 24 K\\\cline{2-2}
%                &RAM 64 K + 16 K de vídeo\\\cline{2-2}
%                &Ampliable a 512 K\\\hline
%     Visualización& \\de Vídeo&24 Lineas de 40caracteres\\\cline{2-2}
%                &16 colores de fondo \\
%                & y primer plano ajustables ind.\\\cline{2-2}
%                &127 caracteres predefinidos\\\cline{2-2}
%                &127 caracteres definido por el usuario\\\hline
%     Interfaces&Cassete\\\cline{2-2}
%                &T.V.\\\cline{2-2}
%                &Monitor Vídeo Compuesto\\\hline
%     Lenguaje& \\suministrado& BASIC, NODDY, ensamblador\\\hline
%     Teclado&79 teclas\\\hline
%     Otros&Conector Hi-Fi\\\hline
%   \end{tabular}
% \end{table}

% Por las especificaciones dadas vemos que es una máquina completa con una apariencia que supera a varios
% ordenadores de la época, esta estaba diseñada para que sea fácil de desarmar y tener  acceso al interior.
% Se da una observación a que su placa base tiene un gran número de circuitos integrados pero sin la
% utilización de algunas \textit{ULA's}\footnote{Uncommited Logic Array} de mayor capacidad en la época,
% haciendo que el mantenimiento correctivo del computador sea más eficiente, ya que las \textit{ULA's}
% dificultan la tarea anterior.

% \subsection{Estructura}

% El \textbf{Memotech MTX 512} contaba con un procesador de 8 bits que viene siendo el
% \textit{Z80A}\footnote{Usados también en MSX, TRS-80 y algunas gamas ZX*}, donde estos procesadores
% tuvieron una gran popularidad en la época de los 80s\ -\ 90s. La memoria principal llega a tener los $64 [KB]$
% expandible a $512 [KB]$. El teclado viene siendo el teclado que conocemos hoy en día descartando las teclas
% direccionales.

% En los periféricos se encontraba una entrada para \textit{Joysticks} que cumplieran con la norma
% \textit{Atari}, dando a entender que tal vez tenia la capacidad de correr videojuegos. También tenía interfaces
% en paralelo \textit{RS232} que era una entrada para el gran número de impresoras.
% Contenía otra interfaz para \textit{Cassetes} y otra para conectar un altavoz de alta fidelidad,
% dicha entrada \textit{HI-FI} no era común para los ordenadores de la época. Asimismo contenía un enchufe
% a monitor.

% Un elemento que hacía destacar esta máquina de otros ordenadores es la posesión de un \textit{ensamblador-
%   desensamblador}\label{asm} que, junto con el paquete de software \textit{Front Panel}, suministra la programación a
% código máquina, aunque venia con poca documentación.

% \subsection{Software}

% Esta computadora personal contenía lenguajes de programación previamente ya suministrados, siendo uno de
% ellos \textbf{NODDY}, conteniendo solo 11 órdenes siendo sencillo para usuarios con necesidad de manejar texto
% ya que este lenguaje no podía efectuar operaciones aritméticas. Igualmente contenía el famoso lenguaje
% \textit{BASIC} y la capacidad de utilizar el lenguaje \textit{Ensamblador}\label{asm}

% Podemos concluir que la \textbf{Memotech MTX 512} era una excelente máquina para usuarios tanto generales como
% programadores por la versatilidad de su arquitectura y su software, aunque la documentación no tenia la
% calidad que se requería, esta no segaba la capacidad de la \textbf{Memotech}. De diseño elegante y fácil
% aprendizaje, la computadora presente contenía lo necesario para el tiempo de trabajar y la hora de recreación.

% %\newpage

% \section{Términos Clave}

% En esta sección de la revista da ciertas definiciones que se usan en el campo
% de la de la computación, tanto hardware como software, en este caso se
% encuentran dos secciones \texttt{Términos Clave}. La primera que encontramos
% habla sobre la Matriz Lógica no Comprometida (ULA), la revista da un análisis de
% la \textbf{ULA}, la que ayuda en la construcción de ordenadores complejos al
% igual que otros dispositivos como el CPU, RAM y ROM.\@

% Páginas más adelante, hay otro articulo perteneciente al nombre de la sección
% ``Términos Clave'', este articulo se titula Autor Original que habla de un
% programa que genere por sí mismo otros programas, que puede ser un compilado,
% más se explica a grandes rasgos en que consiste este articulo.

% \subsection{Diseñada a medida\ -\ ULA}

% Una \textbf{ULA} no es más que un circuito integrado con un gran número de
% compuertas lógicas muy parecida a la programación de la ROM.\@ Algo de destacar
% de este articulo es el énfasis que le dan a las ULAs, ya que estas eran una
% clase de circuitos económicos y de gran potencia para la construcción de los
% ordenadores de aquella época. Las ULAs eran modificadas para la realización de
% las operaciones que necesitaba el diseñador sin llegar a ser comprometidas.\@


% El Artículo de la ULA también abarca una imagen que muestra el diagrama de la
% ULA de manera ``Despiezada'' mostrando las capas de los semiconductores que son
% tratados químicamente de forma individual para crear los elementos del circuito.

% \subsection{Autor Original}

% El articulo habla de los programas que generan programas, que viene siendo un
% \textit{compilador} donde este variaba la capacidad del ordenador, ya que,
% como lo dice la revista, en los ordenadores personales no venía un compilador
% tan robusto que imprimiera en pantalla exactamente el error de compilación,
% simplemente decía algo parecido a \texttt{¿Error de sintaxis?} sin especificar
% la línea y/o la columna en la que se dio el error, estos impedimentos se
% solucionaban comprando más ROM o cartuchos de software que permitiera expandir
% la gamma de ordenes del lenguaje, en este caso \texttt{BASIC}.








  
% \end{multicols} 

\end{document}
