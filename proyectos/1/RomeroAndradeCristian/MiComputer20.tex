\documentclass{article}

\usepackage[letterpaper,right=2cm,left=2cm,top=3cm]{geometry}
\usepackage[utf8]{inputenc}
\usepackage[spanish]{babel}
\usepackage{graphicx}
\usepackage{hyperref}
\usepackage{fancyhdr}
\usepackage{multicol}

\pagestyle{fancy}

\begin{document}

\begin{titlepage}
  \begin{center}
    \Huge{Universidad Nacional Autónoma de México}
    \vfill
    \includegraphics[width=0.2\linewidth]{img/UNAM_INGENIERIA}
    \vfill
    \LARGE{\emph{Facultad de Ingeniería}}
    \vfill
    \LARGE{\texttt{Sistemas Operativos}}
    \vfill
    \Large{Alumno: Romero Andrade Cristian\\Número de Cuenta: 313160282} 
    
    \vfill
    
    \huge{Proyecto 1\\MyComputer $20^o$ Fascículo}
    \vfill
    
    
    \large{Profesor: Gunnar Eyal Wolf Iszaevich}

    \vfill

    {\large \today\par}
    \newpage
  \end{center}
\end{titlepage}

% \section{Resumen}

%\tableofcontents\newpag
e
\section{Introducción}\label{sec:introduction}

El presente consiste sobre una revista famosa de los antaños 80's\footnote{Abril 1984} \texttt{MyComputer},
la versión de habla hispana de la revista \texttt{Home Computer Course} la cual contiene artículos sobre
diversos temas, como son el hardware, software, programas informáticos, programación en \textit{BASIC} y el
tema que siempre destaca en la revista, la revisión y análisis de un \textit{microordenador contemporáneo} donde
exponen imágenes de sus partes internas de manera despiezada.

En este fascículo toma en su sección \textit{Modelos de Hardware} la \textbf{Memotech MTX 512}, un computadora
doméstica de arquitectura \textit{Z80} lanzada por \textit{Memotech} entre 1983 y 1984.
\\
\begin{multicols}{2}

\section{Modelo de Hardware\ -\ Memotech MTX 512}

Toda revista \texttt{MyComputer} tiene esta sección donde hablan de una computadora, exhibiendo sus partes y
componentes, la computadora de la que habla específicamente este fascículo es la \textbf{Memotech MTX 512},
un computador de arquitectura \textit{Z80}. Empezaremos mostrando la lista de especificaciones que nos presenta
la revista.

\begin{table}[h]
  \centering
  \begin{tabular}{|c||c|}
    \hline
    Dimensiones&488x202x56 [mm]\\\hline
    CPU&Z80\\\hline
    Reloj&4 [MHz]\\\hline
    Memoria&ROM: 24 K\\\cline{2-2}
               &RAM 64 K + 16 K de vídeo\\\cline{2-2}
               &Ampliable a 512 K\\\hline
    Visualización& \\de Vídeo&24 Lineas de 40caracteres\\\cline{2-2}
               &16 colores de fondo \\
               & y primer plano ajustables ind.\\\cline{2-2}
               &127 caracteres predefinidos\\\cline{2-2}
               &127 caracteres definido por el usuario\\\hline
    Interfaces&Cassete\\\cline{2-2}
               &T.V.\\\cline{2-2}
               &Monitor Vídeo Compuesto\\\hline
    Lenguaje& \\suministrado& BASIC, NODDY, ensamblador\\\hline
    Teclado&79 teclas\\\hline
    Otros&Conector Hi-Fi\\\hline
  \end{tabular}
\end{table}

Por las especificaciones dadas vemos que es una máquina completa con una apariencia que supera a varios
ordenadores de la época, esta estaba diseñada para que sea fácil de desarmar y tener  acceso al interior.
Se da una observación a que su placa base tiene un gran número de circuitos integrados pero sin la
utilización de algunas \textit{ULA's}\footnote{Uncommited Logic Array} de mayor capacidad en la época,
haciendo que el mantenimiento correctivo del computador sea más eficiente, ya que las \textit{ULA's}
dificultan la tarea anterior.

\subsection{Estructura}

El \textbf{Memotech MTX 512} contaba con un procesador de 8 bits que viene siendo el
\textit{Z80A}\footnote{Usados también en MSX, TRS-80 y algunas gamas ZX*}, donde estos procesadores
tuvieron una gran popularidad en la época de los 80s\ -\ 90s. La memoria principal llega a tener los $64 [KB]$
expandible a $512 [KB]$. El teclado viene siendo el teclado que conocemos hoy en día descartando las teclas
direccionales.

En los periféricos se encontraba una entrada para \textit{Joysticks} que cumplieran con la norma
\textit{Atari}, dando a entender que tal vez tenia la capacidad de correr videojuegos. También tenía interfaces
en paralelo \textit{RS232} que era una entrada para el gran número de impresoras.
Contenía otra interfaz para \textit{Cassetes} y otra para conectar un altavoz de alta fidelidad,
dicha entrada \textit{HI-FI} no era común para los ordenadores de la época. Asimismo contenía un enchufe
a monitor.

Un elemento que hacía destacar esta máquina de otros ordenadores es la posesión de un \textit{ensamblador-
  desensamblador}\label{asm} que, junto con el paquete de software \textit{Front Panel}, suministra la programación a
código máquina, aunque venia con poca documentación.

\subsection{Software}

Esta computadora personal contenía lenguajes de programación previamente ya suministrados, siendo uno de
ellos \textbf{NODDY}, conteniendo solo 11 órdenes siendo sencillo para usuarios con necesidad de manejar texto
ya que este lenguaje no podía efectuar operaciones aritméticas. Igualmente contenía el famoso lenguaje
\textit{BASIC} y la capacidad de utilizar el lenguaje \textit{Ensamblador}\label{asm}

Podemos concluir que la \textbf{Memotech MTX 512} era una excelente máquina para usuarios tanto generales como
programadores por la versatilidad de su arquitectura y su software, aunque la documentación no tenia la
calidad que se requería, esta no segaba la capacidad de la \textbf{Memotech}. De diseño elegante y fácil
aprendizaje, la computadora presente contenía lo necesario para el tiempo de trabajar y la hora de recreación.

\section{Terminos Clave}

En esta sección de la revista da ciertas definiciónes que se usan en el campo
de la de la computación, tanto hardware como software, en este caso se
encunetrasn dos secciones \texttt{Terminos Clave}. La primera que encontramos
habla sobre la Matriz Lógia no Comprometida (ULA), la revista da un análisis de
la \textbf{ULA}, la que ayuda en la contrucción de ordenadores complejos al
igual que otros dispositivos como el CPU, RAM y ROM.\@

Una \textbf{ULA} no es más que un circuito integrado con un gran número de
compuertas lógicas muy parecida a la programación de la ROM.\@ Algo de destacar
de este articuli es el enfacis que le dan a las ULAs, ya que estas erán una
clase de circuitos económicos y de gran potencia para la contrucción de los
ordenadores de aqulla epoca. Las ULAs erán modificadas para la realicación de
las operaciones que necesitaba el diseñador sin llegar a ser comprometidas.\@


El Artículo de la ULA tambien abarca una imagen que muestra el diagrama de la
ULA de manera ``Despiezada'' hablando de los elementos de ella y de como
estan contituidos. 






  
\end{multicols}
\end{document}
